\section{Usage Scenarios}
\label{sec:3}

This section introduces how to use the framework for common usage scenarios. The examples are at the high level, no code is provided. If you want to refer to concrete examples in code, please refer to the package \emph{ch.epfl.ts.example} or check it out on \href{https://github.com/merlinND/TradingSimulation/tree/master/ts/src/main/scala/ch/epfl/ts/example}{Github}.

\subsection{Bitcoin Trading}

In this example, we demonstrate how to use the framework to experiment Bitcoin trading with the \emph{MovingAverageTrader} on live market data. The component graph of this application is as follows:

\noindent
\includegraphics[width=\textwidth]{img/examples/btce}

The BTC-E fetchers get orders and transactions from \url{btc-e.com}, feed them to the market simulator. The market simulator generates transactions based on the bid and ask orders from the trader. The trader sends bid or ask orders to the market simulator based on the transactions received from the market simulator. The printer prints information about the market in the console.

\subsection{Forex Trading}

In this example, we demonstrate how to use the framework to experiment Forex trading with the \emph{MovingAverageTrader} on live Forex data. The component graph of this application is as follows:

\noindent
\includegraphics[width=\textwidth]{img/examples/forex-live}

In the component graph above, the fetcher gets live quotes data from \url{webrates.truefx.com}, and feeds them into the market simulator, trader and broker.

The trader makes sell and buy decisions based on the quotes received from the fetcher, and send them to the broker.

The broker receives orders from the trader, and forward them into the market simulator on behalf of the trader. Note that the usage of broker is optional, it's used here only for illustrating purpose.

The market simulator generates transactions based on the orders received from brokers, and sends the transaction result back to the broker.

\subsection{Replay}

In this example, we demonstrate how to use the framework to experiment Forex trading with the \emph{MovingAverageTrader} on history Forex data. The component graph is as follows:

\noindent
\includegraphics[width=\textwidth]{img/examples/forex-replay}

In the component graph above, the \emph{HistDataCSVFetcher} reads history quotes from CSV files, and send them to the market simulator and trader.

The trader produces ask or bid orders based on the quotes data, and sends them to the market simulator. The market simulator generates transactions based on the orders received from the trader.

The market simulator generates transactions based on received orders, and sends the transactions to the printer, which in turn prints them in the console.

\subsection{Evaluating the Performance of Traders}

This example demonstrates how to evaluate the performance of a trader. For this purpose, we have to use the component \emph{Evaluator}. The \emph{Evaluator} encapsulates a trader component -- that means if you want to connect an upstream component to the trader, you can connect it to the \emph{Evaluator}. And if you want to connect the trader to a downstream component, you connect the \emph{Evaluator} to the downstream component. The wirings inside \emph{Evaluator} will ensure that everything works correctly just like you wire the two components separately.

The component graph is as follows:

\noindent
\includegraphics[width=\textwidth]{img/examples/evaluation}

As it's show in the graph above, the evaluator encapsulates an instance of the \emph{MovingAverageTrader}. They should be considered as a single component, instead of two. It receives transaction and quote data from the market simulator. It sends evaluation report to the printer, and orders to the broker.

The market simulator receives orders from the broker and quotes from the fetcher. It sends the transaction result back to the broker.

The fetcher reads history quotes from file system and sends them to the market simulator. Which fetcher or dataset to use for the evaluation is up to the designer of the experiment.

If you're interested in a code example, you can checkout out at \href{https://github.com/merlinND/TradingSimulation/blob/master/ts/src/main/scala/ch/epfl/ts/evaluation/EvaluationRunner.scala}{Github}.
