\section{Simulation}

\subsection{Initial Parameters}
Even though our project implements essential parts of the foreign exchange 
market, many scenarios that can happen in reality cannot be modelled in out 
simulation. Given the interplay of various counterparts involved and also the
diversity of possible trading options (out of which we only implement spot transactions) make the  market an extremely complex system.

We try to simplify as much as possible by deviding the counterparts into
non-financial traders (that use our strategies) and market maker. There is 
one market maker per symbol and each trader is only trading on one symbol. 

Finding appropriate parameters to configure a simulation that reflects the
true market has become a difficult task since existing statistics focus on 
the effects of the market, meaning the daily turnover, volumes that are
traded or market share distribution. However, it seems impossible to find 
facts like "how many people are involved?" or "what are their funds?".
As such, we can find information about what results from people's trading 
activity i.e. the result of people having money to play with and be active.
However, we need to find a reasonable concept to set their initial funds to allow them being active.

- to initialize the simulation we assign money to traders and market makers in a certain currency (e.g. USD) and they receive the respective amount in the currency they need. we do it the following way:

for each symbol

1. define number of non-financial traders

2. for each trader: assign initial funds as 1-year income according to a local income distribution (e.g. Switzerland: http://www.bfs.admin.ch/bfs/portal/en/index/themen/03/04/blank/key/lohnstruktur/lohnverteilung.html)

3. compute market makers initial funds as sum(funds(all traders)) * scale

\subsection{Price Definition in Extreme Cases}

I might also have found a solution to "how is the price defined when there is nobody on the other side (offering when I want to buy or asking when I want to sell)?".
here http://www.forextraders.com/market-maker-forex-brokers.html I read that each market maker defines its own spread, which I understand the following way:

if there are only sellers and no buyers:
-  the ask price is well defined as the lowest ask offer but the bid price is not
- the market maker defines the bid price by applying its own spread to the lowest ask price
-> selling at market price means selling to the market maker for the lowest ask price MINUS the spread

if there are only buyers and no sellers:
- the bid price is well defined as the highest bid offer but the ask price is not
- again, the market maker defines the bid price by applying its own spread to the highest bid price
-> buying at market price means buying from the market maker at the highest bid price PLUS the spread

\subsection{Brain Storm}

Market Simulation Research

[1] http://www.tpc.org/tpc_documents_current_versions/pdf/tpce-v1.14.0.pdf
[2] www.swissquote.com
[3] http://vantagepointtrading.com/daily-forex-stats
[4] http://www.bis.org/publ/qtrpdf/r_qt1312e.pdf
[5] turnovers by counterparty and currency pairs:
http://www.reuters.com/article/2013/09/05/bis-survey-volumes-idUSL6N0GZ34R20130905

[6] largest forex centers:
http://countingpips.com/fx/2011/08/8-largest-forex-trading-centers-in-the-world/

[7] directional trading volume:
http://www.dailyfx.com/forex/technical/article/special_report/2015/04/13/forex-introducing-volume-by-price.html

[8] https://mahifx.com/blog/50-fascinating-facts-about-forex

[9] largest brokers by volume:
http://www.myfxbook.com/forex-broker-volume

[10] volume survey (North America) with explanation
http://www.newyorkfed.org/fxc/volumesurvey/explanatory_notes.html

[11] forex glossary
http://www.cmsfx.com/en/forex-education/Forex-Glossary/ask-price/

[12] salary distribution switzerland
http://www.bfs.admin.ch/bfs/portal/en/index/themen/03/04/blank/key/lohnstruktur/lohnverteilung.html

[13] explanation of market makers
http://www.forextraders.com/market-maker-forex-brokers.html

p.47 
- Entity Relationships
- Trade Types
- run historical data for 300 business days before simulation starts

Wanted:
- trader initial funds -> half of annual income according to [12]?
- broker initial funds (for leverage trades)
- market maker funds
- number of traders per currency pair
- one market maker per currency pair?

Assumptions:
1. all traders come from the same country
    -> initial funds are distributed according to the local wealth distribution

    2. number of traders trading a particular symbol reflects [10]

    Configuration:
    country:    Switzerland, 5th largest forex center (in 2011), Average Daily Trading Volume: \$263 billion, Percentage of Daily Global Forex Volume: 5 \%
    for each symbol
    1. define number of non-financial traders
    2. assign initial funds to each trader according to [12] equivalent to 1-year income
    3. compute market makers initial funds as sum(funds(all traders)) * scale


    counterparties: pool of non-financial customers, 1 market maker

    instruments:    only spot transactions
